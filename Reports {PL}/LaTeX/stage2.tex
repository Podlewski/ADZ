\documentclass{classrep}
\usepackage[utf8]{inputenc}
\frenchspacing

\usepackage{graphicx}
\usepackage[usenames,dvipsnames]{color}
\usepackage[hidelinks]{hyperref}
\usepackage{float}
\usepackage{setspace}
\usepackage{amsmath, amssymb, mathtools}
\usepackage{subcaption}

\usepackage{booktabs}
\usepackage{graphicx}
\usepackage{pdflscape}

\setlength{\abovecaptionskip}{-10pt}

\studycycle{Informatyka stosowana, studia dzienne, II st.}
\coursesemester{II}
\coursename{Analiza danych złożonych}
\courseyear{2020/2021}

\courseteacher{dr hab. inż. Agnieszka Duraj}
\coursegroup{poniedziałek, 11:45}

\author{%
  \studentinfo{Paweł Galewicz}{234053}\\
  \studentinfo{Karol Podlewski}{234106}%
}

\title{Etap 2}

\begin{document}
\maketitle

\setstretch{1.5}

\tableofcontents
\setstretch{1.25}
\newpage

\section{Cel}

Zadanie polegało na analizie zachowania klasyfikatorów w przypadku wystąpienia dryftu w strumieniu danych dla różnych parametrów początkowych. 

\section{Implementacja}

Program został stworzony w języku Python w wersji 3.8.6, przy wsparciu bibliotek scikit-multiflow oraz scikit w celu skorzystania z algorytmów przeznaczonych do detekcji dryftu oraz klasyfikacji.

Wybranym przez nas klasyfikatorem były algorytmy:
\begin{enumerate}
    \item Naiwny klasyfikatora Bayesa
    \item Maszyna wektorów nośnych
    \item K-najbliższych sąsiadów
    \item Sztuczne sieci neuronowe
\end{enumerate}

Do detekcji dryftów wykorzystaliśmy algorytmy 
\begin{enumerate}
    \item DDM
    \item EDDM
    \item ADWIN
    \item Page-Hinkley
\end{enumerate}

Wykorzystano też bibliotekę argparse - w stworzonym rozwiązaniu w łatwy sposób można określić większość parametrów algorytmów, takich jak liczbę sąsiadów w KNN, rodzaj funkcji jądra w SVM czy maksymalną liczbę iteracji w sieciach neuronowych, a także podział zbioru treningowego oraz sam plik z danymi.

\section{Opis klasyfikatorów}

\subsection{Naiwny klasyfikatora Bayesa}
Naiwny klasyfikatora Bayesa dokonuje klasyfikacji na bazie twierdzenia Bayesa:
$$ P(A \mid B) = \frac{P(B \mid A) \, P(A)}{P(B)} $$
gdzie:
\begin{itemize}
    \item $A$, $B$ -- zdarzenia
    \item $P(A \mid B)$ -- prawdopodobieństwo zdarzenia $A$, o ile zajdzie $B$
    \item $P(B \mid A)$ -- prawdopodobieństwo zdarzenia $B$, o ile zajdzie $A$
    \item $P(A)$ -- prawdopodobieństwo wystąpienia zdarzenia $A$
    \item $P(B)$ -- suma prawdopodobieństw wszystkich potencjalnych skutków zdarzenia: $P(B)=\sum P(B\mid A)P(B)$
\end{itemize}

Model naiwnego klasyfikatora Bayesa zakłada, że dana cecha klasy jest niepowiązana z pozostałymi cechami. Każda z cech indywidualnie wskazuje na prawdopodobieństwo przynależności do danej klasy.
Sprawdza się najlepiej przy dużych zbiorach danych. Jest wykorzystywany m.in. przy filtrowaniu spamu, diagnozie medycznej, czy prognozowaniu pogody.

\subsection{Maszyna wektorów nośnych}
Maszyna wektorów nośnych jest klasyfikatorem liniowym. Algorytm polega na rozdzieleniu obiektów o różnej przynależności klasowej za pomocą hiperpłaszczyzn, które mają być od siebie możliwe jak najbardziej oddalone - taką odległość nazywa się marginesem klasyfikatora, a hiperpłaszczyzny z największym marginesem wektorami nośnymi. 

Algorytm bardzo dobrze sobie radzi z danymi liniowo separowanymi, ale nie zawsze będzie istniała hiperpłaszczyzna rozdzielająca, która zapewni poprawną klasyfikację wszystkich elementów zbioru. W takich przypadkach maszyna wektorów nośnych za pomocą funkcji jądrowych transformuje przestrzeń do postaci liniowo separowanej.

\subsection{Klasyfikator k-najbliższych sąsiadów}
Algorytm k najbliższych sąsiadów jest klasyfikatorem (ściślej algorytmem regresji regresji nieparametrycznej). Algorytm ten zakłada dany zbiór uczący, w którym znajdują się już sklasyfikowane dane. Schemat składa się z szukania \textit{k} obiektów najbliższych do obiektu klasyfikowanego. Następnie, przyporządkowuje się nowy obiekt do najczęściej występującej klasy w obrębie jego k-najbliższych sąsiadów.\\

\subsection{Algorytm sztucznych sieci neuronowych}
Sztuczna siec neuronowa  jest połączeniem wielu elementow nazywanych sztucznymi neuronami, które tworzą conajmniej trzy warstwy: wejściową, ukrytą oraz wyjściową. Neurony przetwarzają informacje dzięki nadaniu im parametrów które nazywane są wagami. Podstawą tworzenia sieci neuronowej jest modyfikowanie współczynnika wagowego połączeń w celu uzyskania poprawnych wyników.

\section{Badania}

Celem przeprowadzonych przez nas badań było sprawdzenie różnych klasyfikatorów pod względem odporności na zmian dryftu na zadanym strumieniu danych. W tym celu stworzyliśmy modele klasyfikatorów z różnymi parametrami startowymi. Zbiór treningowy stanowił 20\% wszystkich danych. Dla wyuczonych modeli generowaliśmy macierz pomyłek i na jej podstawie wyliczaliśmy 4 miary ewaluacyjne:
\begin{itemize}
    \item Dokładność (ang. \textit{Accuracy}) -- określa jak duży odsetek obserwacji został prawidłowo zaklasyfikowany.
    \item Precyzja (ang. \textit{Precision}) -- oznacza jak dużo zaklasyfikowanych do danej klasy obserwacji rzeczywiście do niej należy. 
    \item Czułość (ang. \textit{Sensitivity}) -- stosunek liczby obserwacji oznaczonych jako \textit{true positive} do sumy \textit{true positive} i \textit{false negative}. Jej wartość interpretować można jako zdolność do prawidłowego zakwalifikowania obserwacji do odpowiedniej klasy. 
    \item Specyficzność (ang. \textit{Specificity}) -- stosunek liczby obserwacji \textit{true negative} do sumy \textit{true negative} i \textit{false positive}. Określa jak dobrze model rozpoznaje, że dana obserwacja nie należy do danej klasy.
\end{itemize} 

~\\ Każdy klasyfikator badany był pod względem wpływu parametrów startowych na wystąpienia dryftu. Zjawisko to sprawdzaliśmy wykorzystując kolejne algorytmy detekcji. Wartości miar dla kolejnych próbek, wraz z momentami pojawienia się dryftu zostały zaprezentowane na wykresach.

~\\ Wybrany przez nas zbiór danych to \href{https://www.kaggle.com/jsphyg/weather-dataset-rattle-package}{\textbf{Rain in Australia}}, który zawiera historię danych pogodowych z 10 lat (data, lokalizacja, temperatury, opady, wiatr, ciśnienie, wilgotność, nasłonecznienie itp) wraz z informacją czy następnego dnia padało - jest to cel klasyfikacji dla tego zbioru danych.

~\\ W badaniach uwzględniliśmy następujące konfiguracje klasyfikatorów:
\begin{itemize}
    \item Naiwny klasyfikator Bayesa -- brak parametrów do zmiany.
    \item Maszyna wektorów nośnych -- jądro wielomianowe lub radialne, regularyzacja 0,1 lub 1.
    \item Klasyfikator k-najbliższych sąsiadów -- 2 lub 5 sąsiadów.
    \item Algorytm sztucznych sieci neuronowych -- 1 lub 10 ukrytych warstw, 500 lub 1000 iteracji. 
\end{itemize}

~\\Z badań otrzymaliśmy następujące wyniki:

%%%%%%%%%%%%%%%%%%%%%%%%%%%%%%%%%%%%%%%%%%%%%%%%%%%%%%%%%%%%%%%%%%%%%%%%%%%%%%%%%%%%%%%%%%%%%%%%

\setkeys{Gin}{draft=True}

\subsection{Naiwny klasyfikatora Bayesa}

\begin{figure}[H]
    \centering
    \begin{subfigure}[b]{0.475\textwidth}
        \centering
        \includegraphics[width=\textwidth]{resources/stage2/dokładność/dokładność_Bayes_DDM_.png}
    \end{subfigure}
    \hfill
    \begin{subfigure}[b]{0.475\textwidth}
        \centering
        \includegraphics[width=\textwidth]{resources/stage2/dokładność/dokładność_Bayes_EDDM_.png}
    \end{subfigure}
    \begin{subfigure}[b]{0.475\textwidth}
        \centering
        \includegraphics[width=\textwidth]{resources/stage2/dokładność/dokładność_Bayes_ADWIN_.png}
    \end{subfigure}
    \hfill
    \begin{subfigure}[b]{0.475\textwidth}
        \centering
        \includegraphics[width=\textwidth]{resources/stage2/dokładność/dokładność_Bayes_PageHinkley_.png}
    \end{subfigure}
    \vskip\baselineskip
    \caption{Dokładność dla naiwnego klasyfikatora Bayesa}
\end{figure}

\begin{figure}[H]
    \centering
    \begin{subfigure}[b]{0.475\textwidth}
        \centering
        \includegraphics[width=\textwidth]{resources/stage2/precyzja/precyzja_Bayes_DDM_.png}
    \end{subfigure}
    \hfill
    \begin{subfigure}[b]{0.475\textwidth}
        \centering
        \includegraphics[width=\textwidth]{resources/stage2/precyzja/precyzja_Bayes_EDDM_.png}
    \end{subfigure}
    \begin{subfigure}[b]{0.475\textwidth}
        \centering
        \includegraphics[width=\textwidth]{resources/stage2/precyzja/precyzja_Bayes_ADWIN_.png}
    \end{subfigure}
    \hfill
    \begin{subfigure}[b]{0.475\textwidth}
        \centering
        \includegraphics[width=\textwidth]{resources/stage2/precyzja/precyzja_Bayes_PageHinkley_.png}
    \end{subfigure}
    \vskip\baselineskip
    \caption{Precyzja dla naiwnego klasyfikatora Bayesa}
\end{figure}

\begin{figure}[H]
    \centering
    \begin{subfigure}[b]{0.475\textwidth}
        \centering
        \includegraphics[width=\textwidth]{resources/stage2/czułość/czułość_Bayes_DDM_.png}
    \end{subfigure}
    \hfill
    \begin{subfigure}[b]{0.475\textwidth}
        \centering
        \includegraphics[width=\textwidth]{resources/stage2/czułość/czułość_Bayes_EDDM_.png}
    \end{subfigure}
    \begin{subfigure}[b]{0.475\textwidth}
        \centering
        \includegraphics[width=\textwidth]{resources/stage2/czułość/czułość_Bayes_ADWIN_.png}
    \end{subfigure}
    \hfill
    \begin{subfigure}[b]{0.475\textwidth}
        \centering
        \includegraphics[width=\textwidth]{resources/stage2/czułość/czułość_Bayes_PageHinkley_.png}
    \end{subfigure}
    \vskip\baselineskip
    \caption{Czułość dla naiwnego klasyfikatora Bayesa}
\end{figure}

\begin{figure}[H]
    \centering
    \begin{subfigure}[b]{0.475\textwidth}
        \centering
        \includegraphics[width=\textwidth]{resources/stage2/specyficzność/specyficzność_Bayes_DDM_.png}
    \end{subfigure}
    \hfill
    \begin{subfigure}[b]{0.475\textwidth}
        \centering
        \includegraphics[width=\textwidth]{resources/stage2/specyficzność/specyficzność_Bayes_EDDM_.png}
    \end{subfigure}
    \begin{subfigure}[b]{0.475\textwidth}
        \centering
        \includegraphics[width=\textwidth]{resources/stage2/specyficzność/specyficzność_Bayes_ADWIN_.png}
    \end{subfigure}
    \hfill
    \begin{subfigure}[b]{0.475\textwidth}
        \centering
        \includegraphics[width=\textwidth]{resources/stage2/specyficzność/specyficzność_Bayes_PageHinkley_.png}
    \end{subfigure}
    \vskip\baselineskip
    \caption{Specyficzność dla naiwnego klasyfikatora Bayesa}
\end{figure}


%%%%%%%%%%%%%%%%%%%%%%%%%%%%%%%%%%%%%%%%%%%%%%%%%%%%%%%%%%%%%%%%%%%%%%%%%%%%%%%%%%%%%%%%%%%%%%%%

\setkeys{Gin}{draft=True}

\subsection{Maszyna wektorów nośnych}

\begin{figure}[H]
    \centering
    \begin{subfigure}[b]{0.475\textwidth}
        \centering
        \includegraphics[width=\textwidth]{resources/stage2/dokładność/dokładność_Svm_DDM_poly-0,1.png}
    \end{subfigure}
    \hfill
    \begin{subfigure}[b]{0.475\textwidth}
        \centering
        \includegraphics[width=\textwidth]{resources/stage2/dokładność/dokładność_Svm_EDDM_poly-0,1.png}
    \end{subfigure}
    \begin{subfigure}[b]{0.475\textwidth}
        \centering
        \includegraphics[width=\textwidth]{resources/stage2/dokładność/dokładność_Svm_ADWIN_poly-0,1.png}
    \end{subfigure}
    \hfill
    \begin{subfigure}[b]{0.475\textwidth}
        \centering
        \includegraphics[width=\textwidth]{resources/stage2/dokładność/dokładność_Svm_PageHinkley_poly-0,1.png}
    \end{subfigure}
    \vskip\baselineskip
    \caption{Dokładność dla maszyny wektorów nośnych - metoda wielomianowa, regularyzacja 0.1}
\end{figure}

\begin{figure}[H]
    \centering
    \begin{subfigure}[b]{0.475\textwidth}
        \centering
        \includegraphics[width=\textwidth]{resources/stage2/precyzja/precyzja_Svm_DDM_poly-0,1.png}
    \end{subfigure}
    \hfill
    \begin{subfigure}[b]{0.475\textwidth}
        \centering
        \includegraphics[width=\textwidth]{resources/stage2/precyzja/precyzja_Svm_EDDM_poly-0,1.png}
    \end{subfigure}
    \begin{subfigure}[b]{0.475\textwidth}
        \centering
        \includegraphics[width=\textwidth]{resources/stage2/precyzja/precyzja_Svm_ADWIN_poly-0,1.png}
    \end{subfigure}
    \hfill
    \begin{subfigure}[b]{0.475\textwidth}
        \centering
        \includegraphics[width=\textwidth]{resources/stage2/precyzja/precyzja_Svm_PageHinkley_poly-0,1.png}
    \end{subfigure}
    \vskip\baselineskip
    \caption{Precyzja dla maszyny wektorów nośnych - metoda wielomianowa, regularyzacja 0.1}
\end{figure}

\begin{figure}[H]
    \centering
    \begin{subfigure}[b]{0.475\textwidth}
        \centering
        \includegraphics[width=\textwidth]{resources/stage2/czułość/czułość_Svm_DDM_poly-0,1.png}
    \end{subfigure}
    \hfill
    \begin{subfigure}[b]{0.475\textwidth}
        \centering
        \includegraphics[width=\textwidth]{resources/stage2/czułość/czułość_Svm_EDDM_poly-0,1.png}
    \end{subfigure}
    \begin{subfigure}[b]{0.475\textwidth}
        \centering
        \includegraphics[width=\textwidth]{resources/stage2/czułość/czułość_Svm_ADWIN_poly-0,1.png}
    \end{subfigure}
    \hfill
    \begin{subfigure}[b]{0.475\textwidth}
        \centering
        \includegraphics[width=\textwidth]{resources/stage2/czułość/czułość_Svm_PageHinkley_poly-0,1.png}
    \end{subfigure}
    \vskip\baselineskip
    \caption{Czułość dla maszyny wektorów nośnych - metoda wielomianowa, regularyzacja 0.1}
\end{figure}

\begin{figure}[H]
    \centering
    \begin{subfigure}[b]{0.475\textwidth}
        \centering
        \includegraphics[width=\textwidth]{resources/stage2/specyficzność/specyficzność_Svm_DDM_poly-0,1.png}
    \end{subfigure}
    \hfill
    \begin{subfigure}[b]{0.475\textwidth}
        \centering
        \includegraphics[width=\textwidth]{resources/stage2/specyficzność/specyficzność_Svm_EDDM_poly-0,1.png}
    \end{subfigure}
    \begin{subfigure}[b]{0.475\textwidth}
        \centering
        \includegraphics[width=\textwidth]{resources/stage2/specyficzność/specyficzność_Svm_ADWIN_poly-0,1.png}
    \end{subfigure}
    \hfill
    \begin{subfigure}[b]{0.475\textwidth}
        \centering
        \includegraphics[width=\textwidth]{resources/stage2/specyficzność/specyficzność_Svm_PageHinkley_poly-0,1.png}
    \end{subfigure}
    \vskip\baselineskip
    \caption{Specyficzność dla maszyny wektorów nośnych - metoda wielomianowa, regularyzacja 0.1}
\end{figure}

\begin{figure}[H]
    \centering
    \begin{subfigure}[b]{0.475\textwidth}
        \centering
        \includegraphics[width=\textwidth]{resources/stage2/dokładność/dokładność_Svm_DDM_poly-1,0.png}
    \end{subfigure}
    \hfill
    \begin{subfigure}[b]{0.475\textwidth}
        \centering
        \includegraphics[width=\textwidth]{resources/stage2/dokładność/dokładność_Svm_EDDM_poly-1,0.png}
    \end{subfigure}
    \begin{subfigure}[b]{0.475\textwidth}
        \centering
        \includegraphics[width=\textwidth]{resources/stage2/dokładność/dokładność_Svm_ADWIN_poly-1,0.png}
    \end{subfigure}
    \hfill
    \begin{subfigure}[b]{0.475\textwidth}
        \centering
        \includegraphics[width=\textwidth]{resources/stage2/dokładność/dokładność_Svm_PageHinkley_poly-1,0.png}
    \end{subfigure}
    \vskip\baselineskip
    \caption{Dokładność dla maszyny wektorów nośnych - metoda wielomianowa, regularyzacja 1}
\end{figure}

\begin{figure}[H]
    \centering
    \begin{subfigure}[b]{0.475\textwidth}
        \centering
        \includegraphics[width=\textwidth]{resources/stage2/precyzja/precyzja_Svm_DDM_poly-1,0.png}
    \end{subfigure}
    \hfill
    \begin{subfigure}[b]{0.475\textwidth}
        \centering
        \includegraphics[width=\textwidth]{resources/stage2/precyzja/precyzja_Svm_EDDM_poly-1,0.png}
    \end{subfigure}
    \begin{subfigure}[b]{0.475\textwidth}
        \centering
        \includegraphics[width=\textwidth]{resources/stage2/precyzja/precyzja_Svm_ADWIN_poly-1,0.png}
    \end{subfigure}
    \hfill
    \begin{subfigure}[b]{0.475\textwidth}
        \centering
        \includegraphics[width=\textwidth]{resources/stage2/precyzja/precyzja_Svm_PageHinkley_poly-1,0.png}
    \end{subfigure}
    \vskip\baselineskip
    \caption{Precyzja dla maszyny wektorów nośnych - metoda wielomianowa, regularyzacja 1}
\end{figure}

\begin{figure}[H]
    \centering
    \begin{subfigure}[b]{0.475\textwidth}
        \centering
        \includegraphics[width=\textwidth]{resources/stage2/czułość/czułość_Svm_DDM_poly-1,0.png}
    \end{subfigure}
    \hfill
    \begin{subfigure}[b]{0.475\textwidth}
        \centering
        \includegraphics[width=\textwidth]{resources/stage2/czułość/czułość_Svm_EDDM_poly-1,0.png}
    \end{subfigure}
    \begin{subfigure}[b]{0.475\textwidth}
        \centering
        \includegraphics[width=\textwidth]{resources/stage2/czułość/czułość_Svm_ADWIN_poly-1,0.png}
    \end{subfigure}
    \hfill
    \begin{subfigure}[b]{0.475\textwidth}
        \centering
        \includegraphics[width=\textwidth]{resources/stage2/czułość/czułość_Svm_PageHinkley_poly-1,0.png}
    \end{subfigure}
    \vskip\baselineskip
    \caption{Czułość dla maszyny wektorów nośnych - metoda wielomianowa, regularyzacja 1}
\end{figure}

\begin{figure}[H]
    \centering
    \begin{subfigure}[b]{0.475\textwidth}
        \centering
        \includegraphics[width=\textwidth]{resources/stage2/specyficzność/specyficzność_Svm_DDM_poly-1,0.png}
    \end{subfigure}
    \hfill
    \begin{subfigure}[b]{0.475\textwidth}
        \centering
        \includegraphics[width=\textwidth]{resources/stage2/specyficzność/specyficzność_Svm_EDDM_poly-1,0.png}
    \end{subfigure}
    \begin{subfigure}[b]{0.475\textwidth}
        \centering
        \includegraphics[width=\textwidth]{resources/stage2/specyficzność/specyficzność_Svm_ADWIN_poly-1,0.png}
    \end{subfigure}
    \hfill
    \begin{subfigure}[b]{0.475\textwidth}
        \centering
        \includegraphics[width=\textwidth]{resources/stage2/specyficzność/specyficzność_Svm_PageHinkley_poly-1,0.png}
    \end{subfigure}
    \vskip\baselineskip
    \caption{Specyficzność dla maszyny wektorów nośnych - metoda wielomianowa, regularyzacja 1}
\end{figure}

\begin{figure}[H]
    \centering
    \begin{subfigure}[b]{0.475\textwidth}
        \centering
        \includegraphics[width=\textwidth]{resources/stage2/dokładność/dokładność_Svm_DDM_rbf-0,1.png}
    \end{subfigure}
    \hfill
    \begin{subfigure}[b]{0.475\textwidth}
        \centering
        \includegraphics[width=\textwidth]{resources/stage2/dokładność/dokładność_Svm_EDDM_rbf-0,1.png}
    \end{subfigure}
    \begin{subfigure}[b]{0.475\textwidth}
        \centering
        \includegraphics[width=\textwidth]{resources/stage2/dokładność/dokładność_Svm_ADWIN_rbf-0,1.png}
    \end{subfigure}
    \hfill
    \begin{subfigure}[b]{0.475\textwidth}
        \centering
        \includegraphics[width=\textwidth]{resources/stage2/dokładność/dokładność_Svm_PageHinkley_rbf-0,1.png}
    \end{subfigure}
    \vskip\baselineskip
    \caption{Dokładność dla maszyny wektorów nośnych - metoda radialna, regularyzacja 0.1}
\end{figure}

\begin{figure}[H]
    \centering
    \begin{subfigure}[b]{0.475\textwidth}
        \centering
        \includegraphics[width=\textwidth]{resources/stage2/precyzja/precyzja_Svm_DDM_rbf-0,1.png}
    \end{subfigure}
    \hfill
    \begin{subfigure}[b]{0.475\textwidth}
        \centering
        \includegraphics[width=\textwidth]{resources/stage2/precyzja/precyzja_Svm_EDDM_rbf-0,1.png}
    \end{subfigure}
    \begin{subfigure}[b]{0.475\textwidth}
        \centering
        \includegraphics[width=\textwidth]{resources/stage2/precyzja/precyzja_Svm_ADWIN_rbf-0,1.png}
    \end{subfigure}
    \hfill
    \begin{subfigure}[b]{0.475\textwidth}
        \centering
        \includegraphics[width=\textwidth]{resources/stage2/precyzja/precyzja_Svm_PageHinkley_rbf-0,1.png}
    \end{subfigure}
    \vskip\baselineskip
    \caption{Precyzja dla maszyny wektorów nośnych - metoda radialna, regularyzacja 0.1}
\end{figure}

\begin{figure}[H]
    \centering
    \begin{subfigure}[b]{0.475\textwidth}
        \centering
        \includegraphics[width=\textwidth]{resources/stage2/czułość/czułość_Svm_DDM_rbf-0,1.png}
    \end{subfigure}
    \hfill
    \begin{subfigure}[b]{0.475\textwidth}
        \centering
        \includegraphics[width=\textwidth]{resources/stage2/czułość/czułość_Svm_EDDM_rbf-0,1.png}
    \end{subfigure}
    \begin{subfigure}[b]{0.475\textwidth}
        \centering
        \includegraphics[width=\textwidth]{resources/stage2/czułość/czułość_Svm_ADWIN_rbf-0,1.png}
    \end{subfigure}
    \hfill
    \begin{subfigure}[b]{0.475\textwidth}
        \centering
        \includegraphics[width=\textwidth]{resources/stage2/czułość/czułość_Svm_PageHinkley_rbf-0,1.png}
    \end{subfigure}
    \vskip\baselineskip
    \caption{Czułość dla maszyny wektorów nośnych - metoda radialna, regularyzacja 0.1}
\end{figure}

\begin{figure}[H]
    \centering
    \begin{subfigure}[b]{0.475\textwidth}
        \centering
        \includegraphics[width=\textwidth]{resources/stage2/specyficzność/specyficzność_Svm_DDM_rbf-0,1.png}
    \end{subfigure}
    \hfill
    \begin{subfigure}[b]{0.475\textwidth}
        \centering
        \includegraphics[width=\textwidth]{resources/stage2/specyficzność/specyficzność_Svm_EDDM_rbf-0,1.png}
    \end{subfigure}
    \begin{subfigure}[b]{0.475\textwidth}
        \centering
        \includegraphics[width=\textwidth]{resources/stage2/specyficzność/specyficzność_Svm_ADWIN_rbf-0,1.png}
    \end{subfigure}
    \hfill
    \begin{subfigure}[b]{0.475\textwidth}
        \centering
        \includegraphics[width=\textwidth]{resources/stage2/specyficzność/specyficzność_Svm_PageHinkley_rbf-0,1.png}
    \end{subfigure}
    \vskip\baselineskip
    \caption{Specyficzność dla maszyny wektorów nośnych - metoda radialna, regularyzacja 0.1}
\end{figure}

\begin{figure}[H]
    \centering
    \begin{subfigure}[b]{0.475\textwidth}
        \centering
        \includegraphics[width=\textwidth]{resources/stage2/dokładność/dokładność_Svm_DDM_rbf-1,0.png}
    \end{subfigure}
    \hfill
    \begin{subfigure}[b]{0.475\textwidth}
        \centering
        \includegraphics[width=\textwidth]{resources/stage2/dokładność/dokładność_Svm_EDDM_rbf-1,0.png}
    \end{subfigure}
    \begin{subfigure}[b]{0.475\textwidth}
        \centering
        \includegraphics[width=\textwidth]{resources/stage2/dokładność/dokładność_Svm_ADWIN_rbf-1,0.png}
    \end{subfigure}
    \hfill
    \begin{subfigure}[b]{0.475\textwidth}
        \centering
        \includegraphics[width=\textwidth]{resources/stage2/dokładność/dokładność_Svm_PageHinkley_rbf-1,0.png}
    \end{subfigure}
    \vskip\baselineskip
    \caption{Dokładność dla maszyny wektorów nośnych - metoda radialna, regularyzacja 1}
\end{figure}

\begin{figure}[H]
    \centering
    \begin{subfigure}[b]{0.475\textwidth}
        \centering
        \includegraphics[width=\textwidth]{resources/stage2/precyzja/precyzja_Svm_DDM_rbf-1,0.png}
    \end{subfigure}
    \hfill
    \begin{subfigure}[b]{0.475\textwidth}
        \centering
        \includegraphics[width=\textwidth]{resources/stage2/precyzja/precyzja_Svm_EDDM_rbf-1,0.png}
    \end{subfigure}
    \begin{subfigure}[b]{0.475\textwidth}
        \centering
        \includegraphics[width=\textwidth]{resources/stage2/precyzja/precyzja_Svm_ADWIN_rbf-1,0.png}
    \end{subfigure}
    \hfill
    \begin{subfigure}[b]{0.475\textwidth}
        \centering
        \includegraphics[width=\textwidth]{resources/stage2/precyzja/precyzja_Svm_PageHinkley_rbf-1,0.png}
    \end{subfigure}
    \vskip\baselineskip
    \caption{Precyzja dla maszyny wektorów nośnych - metoda radialna, regularyzacja 1}
\end{figure}

\begin{figure}[H]
    \centering
    \begin{subfigure}[b]{0.475\textwidth}
        \centering
        \includegraphics[width=\textwidth]{resources/stage2/czułość/czułość_Svm_DDM_rbf-1,0.png}
    \end{subfigure}
    \hfill
    \begin{subfigure}[b]{0.475\textwidth}
        \centering
        \includegraphics[width=\textwidth]{resources/stage2/czułość/czułość_Svm_EDDM_rbf-1,0.png}
    \end{subfigure}
    \begin{subfigure}[b]{0.475\textwidth}
        \centering
        \includegraphics[width=\textwidth]{resources/stage2/czułość/czułość_Svm_ADWIN_rbf-1,0.png}
    \end{subfigure}
    \hfill
    \begin{subfigure}[b]{0.475\textwidth}
        \centering
        \includegraphics[width=\textwidth]{resources/stage2/czułość/czułość_Svm_PageHinkley_rbf-1,0.png}
    \end{subfigure}
    \vskip\baselineskip
    \caption{Czułość dla maszyny wektorów nośnych - metoda radialna, regularyzacja 1}
\end{figure}

\begin{figure}[H]
    \centering
    \begin{subfigure}[b]{0.475\textwidth}
        \centering
        \includegraphics[width=\textwidth]{resources/stage2/specyficzność/specyficzność_Svm_DDM_rbf-1,0.png}
    \end{subfigure}
    \hfill
    \begin{subfigure}[b]{0.475\textwidth}
        \centering
        \includegraphics[width=\textwidth]{resources/stage2/specyficzność/specyficzność_Svm_EDDM_rbf-1,0.png}
    \end{subfigure}
    \begin{subfigure}[b]{0.475\textwidth}
        \centering
        \includegraphics[width=\textwidth]{resources/stage2/specyficzność/specyficzność_Svm_ADWIN_rbf-1,0.png}
    \end{subfigure}
    \hfill
    \begin{subfigure}[b]{0.475\textwidth}
        \centering
        \includegraphics[width=\textwidth]{resources/stage2/specyficzność/specyficzność_Svm_PageHinkley_rbf-1,0.png}
    \end{subfigure}
    \vskip\baselineskip
    \caption{Specyficzność dla maszyny wektorów nośnych - metoda radialna, regularyzacja 1}
\end{figure}

%%%%%%%%%%%%%%%%%%%%%%%%%%%%%%%%%%%%%%%%%%%%%%%%%%%%%%%%%%%%%%%%%%%%%%%%%%%%%%%%%%%%%%%%%%%%%%%%

\setkeys{Gin}{draft=True}

\subsection{Klasyfikator k-najbliższych sąsiadów}

\begin{figure}[H]
    \centering
    \begin{subfigure}[b]{0.475\textwidth}
        \centering
        \includegraphics[width=\textwidth]{resources/stage2/dokładność/dokładność_Knn_DDM_2.png}
    \end{subfigure}
    \hfill
    \begin{subfigure}[b]{0.475\textwidth}
        \centering
        \includegraphics[width=\textwidth]{resources/stage2/dokładność/dokładność_Knn_EDDM_2.png}
    \end{subfigure}
    \begin{subfigure}[b]{0.475\textwidth}
        \centering
        \includegraphics[width=\textwidth]{resources/stage2/dokładność/dokładność_Knn_ADWIN_2.png}
    \end{subfigure}
    \hfill
    \begin{subfigure}[b]{0.475\textwidth}
        \centering
        \includegraphics[width=\textwidth]{resources/stage2/dokładność/dokładność_Knn_PageHinkley_2.png}
    \end{subfigure}
    \vskip\baselineskip
    \caption{Dokładność dla klasyfikatora k-najbliższych sąsiadów - 2 sąsiadów}
\end{figure}

\begin{figure}[H]
    \centering
    \begin{subfigure}[b]{0.475\textwidth}
        \centering
        \includegraphics[width=\textwidth]{resources/stage2/precyzja/precyzja_Knn_DDM_2.png}
    \end{subfigure}
    \hfill
    \begin{subfigure}[b]{0.475\textwidth}
        \centering
        \includegraphics[width=\textwidth]{resources/stage2/precyzja/precyzja_Knn_EDDM_2.png}
    \end{subfigure}
    \begin{subfigure}[b]{0.475\textwidth}
        \centering
        \includegraphics[width=\textwidth]{resources/stage2/precyzja/precyzja_Knn_ADWIN_2.png}
    \end{subfigure}
    \hfill
    \begin{subfigure}[b]{0.475\textwidth}
        \centering
        \includegraphics[width=\textwidth]{resources/stage2/precyzja/precyzja_Knn_PageHinkley_2.png}
    \end{subfigure}
    \vskip\baselineskip
    \caption{Precyzja dla klasyfikatora k-najbliższych sąsiadów - 2 sąsiadów}
\end{figure}

\begin{figure}[H]
    \centering
    \begin{subfigure}[b]{0.475\textwidth}
        \centering
        \includegraphics[width=\textwidth]{resources/stage2/czułość/czułość_Knn_DDM_2.png}
    \end{subfigure}
    \hfill
    \begin{subfigure}[b]{0.475\textwidth}
        \centering
        \includegraphics[width=\textwidth]{resources/stage2/czułość/czułość_Knn_EDDM_2.png}
    \end{subfigure}
    \begin{subfigure}[b]{0.475\textwidth}
        \centering
        \includegraphics[width=\textwidth]{resources/stage2/czułość/czułość_Knn_ADWIN_2.png}
    \end{subfigure}
    \hfill
    \begin{subfigure}[b]{0.475\textwidth}
        \centering
        \includegraphics[width=\textwidth]{resources/stage2/czułość/czułość_Knn_PageHinkley_2.png}
    \end{subfigure}
    \vskip\baselineskip
    \caption{Czułość dla klasyfikatora k-najbliższych sąsiadów - 2 sąsiadów}
\end{figure}

\begin{figure}[H]
    \centering
    \begin{subfigure}[b]{0.475\textwidth}
        \centering
        \includegraphics[width=\textwidth]{resources/stage2/specyficzność/specyficzność_Knn_DDM_2.png}
    \end{subfigure}
    \hfill
    \begin{subfigure}[b]{0.475\textwidth}
        \centering
        \includegraphics[width=\textwidth]{resources/stage2/specyficzność/specyficzność_Knn_EDDM_2.png}
    \end{subfigure}
    \begin{subfigure}[b]{0.475\textwidth}
        \centering
        \includegraphics[width=\textwidth]{resources/stage2/specyficzność/specyficzność_Knn_ADWIN_2.png}
    \end{subfigure}
    \hfill
    \begin{subfigure}[b]{0.475\textwidth}
        \centering
        \includegraphics[width=\textwidth]{resources/stage2/specyficzność/specyficzność_Knn_PageHinkley_2.png}
    \end{subfigure}
    \vskip\baselineskip
    \caption{Specyficzność dla klasyfikatora k-najbliższych sąsiadów - 2 sąsiadów}
\end{figure}


\begin{figure}[H]
    \centering
    \begin{subfigure}[b]{0.475\textwidth}
        \centering
        \includegraphics[width=\textwidth]{resources/stage2/dokładność/dokładność_Knn_DDM_5.png}
    \end{subfigure}
    \hfill
    \begin{subfigure}[b]{0.475\textwidth}
        \centering
        \includegraphics[width=\textwidth]{resources/stage2/dokładność/dokładność_Knn_EDDM_5.png}
    \end{subfigure}
    \begin{subfigure}[b]{0.475\textwidth}
        \centering
        \includegraphics[width=\textwidth]{resources/stage2/dokładność/dokładność_Knn_ADWIN_5.png}
    \end{subfigure}
    \hfill
    \begin{subfigure}[b]{0.475\textwidth}
        \centering
        \includegraphics[width=\textwidth]{resources/stage2/dokładność/dokładność_Knn_PageHinkley_5.png}
    \end{subfigure}
    \vskip\baselineskip
    \caption{Dokładność dla klasyfikatora k-najbliższych sąsiadów - 5 sąsiadów}
\end{figure}

\begin{figure}[H]
    \centering
    \begin{subfigure}[b]{0.475\textwidth}
        \centering
        \includegraphics[width=\textwidth]{resources/stage2/precyzja/precyzja_Knn_DDM_5.png}
    \end{subfigure}
    \hfill
    \begin{subfigure}[b]{0.475\textwidth}
        \centering
        \includegraphics[width=\textwidth]{resources/stage2/precyzja/precyzja_Knn_EDDM_5.png}
    \end{subfigure}
    \begin{subfigure}[b]{0.475\textwidth}
        \centering
        \includegraphics[width=\textwidth]{resources/stage2/precyzja/precyzja_Knn_ADWIN_5.png}
    \end{subfigure}
    \hfill
    \begin{subfigure}[b]{0.475\textwidth}
        \centering
        \includegraphics[width=\textwidth]{resources/stage2/precyzja/precyzja_Knn_PageHinkley_5.png}
    \end{subfigure}
    \vskip\baselineskip
    \caption{Precyzja dla klasyfikatora k-najbliższych sąsiadów - 5 sąsiadów}
\end{figure}

\begin{figure}[H]
    \centering
    \begin{subfigure}[b]{0.475\textwidth}
        \centering
        \includegraphics[width=\textwidth]{resources/stage2/czułość/czułość_Knn_DDM_5.png}
    \end{subfigure}
    \hfill
    \begin{subfigure}[b]{0.475\textwidth}
        \centering
        \includegraphics[width=\textwidth]{resources/stage2/czułość/czułość_Knn_EDDM_5.png}
    \end{subfigure}
    \begin{subfigure}[b]{0.475\textwidth}
        \centering
        \includegraphics[width=\textwidth]{resources/stage2/czułość/czułość_Knn_ADWIN_5.png}
    \end{subfigure}
    \hfill
    \begin{subfigure}[b]{0.475\textwidth}
        \centering
        \includegraphics[width=\textwidth]{resources/stage2/czułość/czułość_Knn_PageHinkley_5.png}
    \end{subfigure}
    \vskip\baselineskip
    \caption{Czułość dla klasyfikatora k-najbliższych sąsiadów - 5 sąsiadów}
\end{figure}

\begin{figure}[H]
    \centering
    \begin{subfigure}[b]{0.475\textwidth}
        \centering
        \includegraphics[width=\textwidth]{resources/stage2/specyficzność/specyficzność_Knn_DDM_5.png}
    \end{subfigure}
    \hfill
    \begin{subfigure}[b]{0.475\textwidth}
        \centering
        \includegraphics[width=\textwidth]{resources/stage2/specyficzność/specyficzność_Knn_EDDM_5.png}
    \end{subfigure}
    \begin{subfigure}[b]{0.475\textwidth}
        \centering
        \includegraphics[width=\textwidth]{resources/stage2/specyficzność/specyficzność_Knn_ADWIN_5.png}
    \end{subfigure}
    \hfill
    \begin{subfigure}[b]{0.475\textwidth}
        \centering
        \includegraphics[width=\textwidth]{resources/stage2/specyficzność/specyficzność_Knn_PageHinkley_5.png}
    \end{subfigure}
    \vskip\baselineskip
    \caption{Specyficzność dla klasyfikatora k-najbliższych sąsiadów - 5 sąsiadów}
\end{figure}

%%%%%%%%%%%%%%%%%%%%%%%%%%%%%%%%%%%%%%%%%%%%%%%%%%%%%%%%%%%%%%%%%%%%%%%%%%%%%%%%%%%%%%%%%%%%%%%%

\setkeys{Gin}{draft=True}

\subsection{Algorytm sztucznych sieci neuronowych}

\begin{figure}[H]
    \centering
    \begin{subfigure}[b]{0.475\textwidth}
        \centering
        \includegraphics[width=\textwidth]{resources/stage2/dokładność/dokładność_MLP_DDM_1-500.png}
    \end{subfigure}
    \hfill
    \begin{subfigure}[b]{0.475\textwidth}
        \centering
        \includegraphics[width=\textwidth]{resources/stage2/dokładność/dokładność_MLP_EDDM_1-500.png}
    \end{subfigure}
    \begin{subfigure}[b]{0.475\textwidth}
        \centering
        \includegraphics[width=\textwidth]{resources/stage2/dokładność/dokładność_MLP_ADWIN_1-500.png}
    \end{subfigure}
    \hfill
    \begin{subfigure}[b]{0.475\textwidth}
        \centering
        \includegraphics[width=\textwidth]{resources/stage2/dokładność/dokładność_MLP_PageHinkley_1-500.png}
    \end{subfigure}
    \vskip\baselineskip
    \caption{Dokładność sztucznej sieci neuronowej - 1 ukryta warstwa, 500 iteracji}
\end{figure}

\begin{figure}[H]
    \centering
    \begin{subfigure}[b]{0.475\textwidth}
        \centering
        \includegraphics[width=\textwidth]{resources/stage2/precyzja/precyzja_MLP_DDM_1-500.png}
    \end{subfigure}
    \hfill
    \begin{subfigure}[b]{0.475\textwidth}
        \centering
        \includegraphics[width=\textwidth]{resources/stage2/precyzja/precyzja_MLP_EDDM_1-500.png}
    \end{subfigure}
    \begin{subfigure}[b]{0.475\textwidth}
        \centering
        \includegraphics[width=\textwidth]{resources/stage2/precyzja/precyzja_MLP_ADWIN_1-500.png}
    \end{subfigure}
    \hfill
    \begin{subfigure}[b]{0.475\textwidth}
        \centering
        \includegraphics[width=\textwidth]{resources/stage2/precyzja/precyzja_MLP_PageHinkley_1-500.png}
    \end{subfigure}
    \vskip\baselineskip
    \caption{Precyzja sztucznej sieci neuronowej - 1 ukryta warstwa, 500 iteracji}
\end{figure}

\begin{figure}[H]
    \centering
    \begin{subfigure}[b]{0.475\textwidth}
        \centering
        \includegraphics[width=\textwidth]{resources/stage2/czułość/czułość_MLP_DDM_1-500.png}
    \end{subfigure}
    \hfill
    \begin{subfigure}[b]{0.475\textwidth}
        \centering
        \includegraphics[width=\textwidth]{resources/stage2/czułość/czułość_MLP_EDDM_1-500.png}
    \end{subfigure}
    \begin{subfigure}[b]{0.475\textwidth}
        \centering
        \includegraphics[width=\textwidth]{resources/stage2/czułość/czułość_MLP_ADWIN_1-500.png}
    \end{subfigure}
    \hfill
    \begin{subfigure}[b]{0.475\textwidth}
        \centering
        \includegraphics[width=\textwidth]{resources/stage2/czułość/czułość_MLP_PageHinkley_1-500.png}
    \end{subfigure}
    \vskip\baselineskip
    \caption{Czułość sztucznej sieci neuronowej - 1 ukryta warstwa, 500 iteracji}
\end{figure}

\begin{figure}[H]
    \centering
    \begin{subfigure}[b]{0.475\textwidth}
        \centering
        \includegraphics[width=\textwidth]{resources/stage2/specyficzność/specyficzność_MLP_DDM_1-500.png}
    \end{subfigure}
    \hfill
    \begin{subfigure}[b]{0.475\textwidth}
        \centering
        \includegraphics[width=\textwidth]{resources/stage2/specyficzność/specyficzność_MLP_EDDM_1-500.png}
    \end{subfigure}
    \begin{subfigure}[b]{0.475\textwidth}
        \centering
        \includegraphics[width=\textwidth]{resources/stage2/specyficzność/specyficzność_MLP_ADWIN_1-500.png}
    \end{subfigure}
    \hfill
    \begin{subfigure}[b]{0.475\textwidth}
        \centering
        \includegraphics[width=\textwidth]{resources/stage2/specyficzność/specyficzność_MLP_PageHinkley_1-500.png}
    \end{subfigure}
    \vskip\baselineskip
    \caption{Specyficzność sztucznej sieci neuronowej - 1 ukryta warstwa, 500 iteracji}
\end{figure}


\begin{figure}[H]
    \centering
    \begin{subfigure}[b]{0.475\textwidth}
        \centering
        \includegraphics[width=\textwidth]{resources/stage2/dokładność/dokładność_MLP_DDM_1-1000.png}
    \end{subfigure}
    \hfill
    \begin{subfigure}[b]{0.475\textwidth}
        \centering
        \includegraphics[width=\textwidth]{resources/stage2/dokładność/dokładność_MLP_EDDM_1-1000.png}
    \end{subfigure}
    \begin{subfigure}[b]{0.475\textwidth}
        \centering
        \includegraphics[width=\textwidth]{resources/stage2/dokładność/dokładność_MLP_ADWIN_1-1000.png}
    \end{subfigure}
    \hfill
    \begin{subfigure}[b]{0.475\textwidth}
        \centering
        \includegraphics[width=\textwidth]{resources/stage2/dokładność/dokładność_MLP_PageHinkley_1-1000.png}
    \end{subfigure}
    \vskip\baselineskip
    \caption{Dokładność sztucznej sieci neuronowej - 1 ukryta warstwa, 1000 iteracji}
\end{figure}

\begin{figure}[H]
    \centering
    \begin{subfigure}[b]{0.475\textwidth}
        \centering
        \includegraphics[width=\textwidth]{resources/stage2/precyzja/precyzja_MLP_DDM_1-1000.png}
    \end{subfigure}
    \hfill
    \begin{subfigure}[b]{0.475\textwidth}
        \centering
        \includegraphics[width=\textwidth]{resources/stage2/precyzja/precyzja_MLP_EDDM_1-1000.png}
    \end{subfigure}
    \begin{subfigure}[b]{0.475\textwidth}
        \centering
        \includegraphics[width=\textwidth]{resources/stage2/precyzja/precyzja_MLP_ADWIN_1-1000.png}
    \end{subfigure}
    \hfill
    \begin{subfigure}[b]{0.475\textwidth}
        \centering
        \includegraphics[width=\textwidth]{resources/stage2/precyzja/precyzja_MLP_PageHinkley_1-1000.png}
    \end{subfigure}
    \vskip\baselineskip
    \caption{Precyzja sztucznej sieci neuronowej - 1 ukryta warstwa, 1000 iteracji}
\end{figure}

\begin{figure}[H]
    \centering
    \begin{subfigure}[b]{0.475\textwidth}
        \centering
        \includegraphics[width=\textwidth]{resources/stage2/czułość/czułość_MLP_DDM_1-1000.png}
    \end{subfigure}
    \hfill
    \begin{subfigure}[b]{0.475\textwidth}
        \centering
        \includegraphics[width=\textwidth]{resources/stage2/czułość/czułość_MLP_EDDM_1-1000.png}
    \end{subfigure}
    \begin{subfigure}[b]{0.475\textwidth}
        \centering
        \includegraphics[width=\textwidth]{resources/stage2/czułość/czułość_MLP_ADWIN_1-1000.png}
    \end{subfigure}
    \hfill
    \begin{subfigure}[b]{0.475\textwidth}
        \centering
        \includegraphics[width=\textwidth]{resources/stage2/czułość/czułość_MLP_PageHinkley_1-1000.png}
    \end{subfigure}
    \vskip\baselineskip
    \caption{Czułość sztucznej sieci neuronowej - 1 ukryta warstwa, 1000 iteracji}
\end{figure}

\begin{figure}[H]
    \centering
    \begin{subfigure}[b]{0.475\textwidth}
        \centering
        \includegraphics[width=\textwidth]{resources/stage2/specyficzność/specyficzność_MLP_DDM_1-1000.png}
    \end{subfigure}
    \hfill
    \begin{subfigure}[b]{0.475\textwidth}
        \centering
        \includegraphics[width=\textwidth]{resources/stage2/specyficzność/specyficzność_MLP_EDDM_1-1000.png}
    \end{subfigure}
    \begin{subfigure}[b]{0.475\textwidth}
        \centering
        \includegraphics[width=\textwidth]{resources/stage2/specyficzność/specyficzność_MLP_ADWIN_1-1000.png}
    \end{subfigure}
    \hfill
    \begin{subfigure}[b]{0.475\textwidth}
        \centering
        \includegraphics[width=\textwidth]{resources/stage2/specyficzność/specyficzność_MLP_PageHinkley_1-1000.png}
    \end{subfigure}
    \vskip\baselineskip
    \caption{Specyficzność sztucznej sieci neuronowej - 1 ukryta warstwa, 1000 iteracji}
\end{figure}

\begin{figure}[H]
    \centering
    \begin{subfigure}[b]{0.475\textwidth}
        \centering
        \includegraphics[width=\textwidth]{resources/stage2/dokładność/dokładność_MLP_DDM_10-500.png}
    \end{subfigure}
    \hfill
    \begin{subfigure}[b]{0.475\textwidth}
        \centering
        \includegraphics[width=\textwidth]{resources/stage2/dokładność/dokładność_MLP_EDDM_10-500.png}
    \end{subfigure}
    \begin{subfigure}[b]{0.475\textwidth}
        \centering
        \includegraphics[width=\textwidth]{resources/stage2/dokładność/dokładność_MLP_ADWIN_10-500.png}
    \end{subfigure}
    \hfill
    \begin{subfigure}[b]{0.475\textwidth}
        \centering
        \includegraphics[width=\textwidth]{resources/stage2/dokładność/dokładność_MLP_PageHinkley_10-500.png}
    \end{subfigure}
    \vskip\baselineskip
    \caption{Dokładność sztucznej sieci neuronowej - 10 ukrytych warstw, 500 iteracji}
\end{figure}

\begin{figure}[H]
    \centering
    \begin{subfigure}[b]{0.475\textwidth}
        \centering
        \includegraphics[width=\textwidth]{resources/stage2/precyzja/precyzja_MLP_DDM_10-500.png}
    \end{subfigure}
    \hfill
    \begin{subfigure}[b]{0.475\textwidth}
        \centering
        \includegraphics[width=\textwidth]{resources/stage2/precyzja/precyzja_MLP_EDDM_10-500.png}
    \end{subfigure}
    \begin{subfigure}[b]{0.475\textwidth}
        \centering
        \includegraphics[width=\textwidth]{resources/stage2/precyzja/precyzja_MLP_ADWIN_10-500.png}
    \end{subfigure}
    \hfill
    \begin{subfigure}[b]{0.475\textwidth}
        \centering
        \includegraphics[width=\textwidth]{resources/stage2/precyzja/precyzja_MLP_PageHinkley_10-500.png}
    \end{subfigure}
    \vskip\baselineskip
    \caption{Precyzja sztucznej sieci neuronowej - 10 ukrytych warstw, 500 iteracji}
\end{figure}

\begin{figure}[H]
    \centering
    \begin{subfigure}[b]{0.475\textwidth}
        \centering
        \includegraphics[width=\textwidth]{resources/stage2/czułość/czułość_MLP_DDM_10-500.png}
    \end{subfigure}
    \hfill
    \begin{subfigure}[b]{0.475\textwidth}
        \centering
        \includegraphics[width=\textwidth]{resources/stage2/czułość/czułość_MLP_EDDM_10-500.png}
    \end{subfigure}
    \begin{subfigure}[b]{0.475\textwidth}
        \centering
        \includegraphics[width=\textwidth]{resources/stage2/czułość/czułość_MLP_ADWIN_10-500.png}
    \end{subfigure}
    \hfill
    \begin{subfigure}[b]{0.475\textwidth}
        \centering
        \includegraphics[width=\textwidth]{resources/stage2/czułość/czułość_MLP_PageHinkley_10-500.png}
    \end{subfigure}
    \vskip\baselineskip
    \caption{Czułość sztucznej sieci neuronowej - 10 ukrytych warstw, 500 iteracji}
\end{figure}

\begin{figure}[H]
    \centering
    \begin{subfigure}[b]{0.475\textwidth}
        \centering
        \includegraphics[width=\textwidth]{resources/stage2/specyficzność/specyficzność_MLP_DDM_10-500.png}
    \end{subfigure}
    \hfill
    \begin{subfigure}[b]{0.475\textwidth}
        \centering
        \includegraphics[width=\textwidth]{resources/stage2/specyficzność/specyficzność_MLP_EDDM_10-500.png}
    \end{subfigure}
    \begin{subfigure}[b]{0.475\textwidth}
        \centering
        \includegraphics[width=\textwidth]{resources/stage2/specyficzność/specyficzność_MLP_ADWIN_10-500.png}
    \end{subfigure}
    \hfill
    \begin{subfigure}[b]{0.475\textwidth}
        \centering
        \includegraphics[width=\textwidth]{resources/stage2/specyficzność/specyficzność_MLP_PageHinkley_10-500.png}
    \end{subfigure}
    \vskip\baselineskip
    \caption{Specyficzność sztucznej sieci neuronowej - 10 ukrytych warstw, 500 iteracji}
\end{figure}


\begin{figure}[H]
    \centering
    \begin{subfigure}[b]{0.475\textwidth}
        \centering
        \includegraphics[width=\textwidth]{resources/stage2/dokładność/dokładność_MLP_DDM_10-1000.png}
    \end{subfigure}
    \hfill
    \begin{subfigure}[b]{0.475\textwidth}
        \centering
        \includegraphics[width=\textwidth]{resources/stage2/dokładność/dokładność_MLP_EDDM_10-1000.png}
    \end{subfigure}
    \begin{subfigure}[b]{0.475\textwidth}
        \centering
        \includegraphics[width=\textwidth]{resources/stage2/dokładność/dokładność_MLP_ADWIN_10-1000.png}
    \end{subfigure}
    \hfill
    \begin{subfigure}[b]{0.475\textwidth}
        \centering
        \includegraphics[width=\textwidth]{resources/stage2/dokładność/dokładność_MLP_PageHinkley_10-1000.png}
    \end{subfigure}
    \vskip\baselineskip
    \caption{Dokładność sztucznej sieci neuronowej - 10 ukrytych warstw, 1000 iteracji}
\end{figure}

\begin{figure}[H]
    \centering
    \begin{subfigure}[b]{0.475\textwidth}
        \centering
        \includegraphics[width=\textwidth]{resources/stage2/precyzja/precyzja_MLP_DDM_10-1000.png}
    \end{subfigure}
    \hfill
    \begin{subfigure}[b]{0.475\textwidth}
        \centering
        \includegraphics[width=\textwidth]{resources/stage2/precyzja/precyzja_MLP_EDDM_10-1000.png}
    \end{subfigure}
    \begin{subfigure}[b]{0.475\textwidth}
        \centering
        \includegraphics[width=\textwidth]{resources/stage2/precyzja/precyzja_MLP_ADWIN_10-1000.png}
    \end{subfigure}
    \hfill
    \begin{subfigure}[b]{0.475\textwidth}
        \centering
        \includegraphics[width=\textwidth]{resources/stage2/precyzja/precyzja_MLP_PageHinkley_10-1000.png}
    \end{subfigure}
    \vskip\baselineskip
    \caption{Precyzja sztucznej sieci neuronowej - 10 ukrytych warstw, 1000 iteracji}
\end{figure}

\begin{figure}[H]
    \centering
    \begin{subfigure}[b]{0.475\textwidth}
        \centering
        \includegraphics[width=\textwidth]{resources/stage2/czułość/czułość_MLP_DDM_10-1000.png}
    \end{subfigure}
    \hfill
    \begin{subfigure}[b]{0.475\textwidth}
        \centering
        \includegraphics[width=\textwidth]{resources/stage2/czułość/czułość_MLP_EDDM_10-1000.png}
    \end{subfigure}
    \begin{subfigure}[b]{0.475\textwidth}
        \centering
        \includegraphics[width=\textwidth]{resources/stage2/czułość/czułość_MLP_ADWIN_10-1000.png}
    \end{subfigure}
    \hfill
    \begin{subfigure}[b]{0.475\textwidth}
        \centering
        \includegraphics[width=\textwidth]{resources/stage2/czułość/czułość_MLP_PageHinkley_10-1000.png}
    \end{subfigure}
    \vskip\baselineskip
    \caption{Czułość sztucznej sieci neuronowej - 10 ukrytych warstw, 1000 iteracji}
    \label{nn10-1000}
\end{figure}

\begin{figure}[H]
    \centering
    \begin{subfigure}[b]{0.475\textwidth}
        \centering
        \includegraphics[width=\textwidth]{resources/stage2/specyficzność/specyficzność_MLP_DDM_10-1000.png}
    \end{subfigure}
    \hfill
    \begin{subfigure}[b]{0.475\textwidth}
        \centering
        \includegraphics[width=\textwidth]{resources/stage2/specyficzność/specyficzność_MLP_EDDM_10-1000.png}
    \end{subfigure}
    \begin{subfigure}[b]{0.475\textwidth}
        \centering
        \includegraphics[width=\textwidth]{resources/stage2/specyficzność/specyficzność_MLP_ADWIN_10-1000.png}
    \end{subfigure}
    \hfill
    \begin{subfigure}[b]{0.475\textwidth}
        \centering
        \includegraphics[width=\textwidth]{resources/stage2/specyficzność/specyficzność_MLP_PageHinkley_10-1000.png}
    \end{subfigure}
    \vskip\baselineskip
    \caption{Specyficzność sztucznej sieci neuronowej - 10 ukrytych warstw, 1000 iteracji}
\end{figure}

\newpage
\section{Wnioski}

\begin{enumerate}
    \item Algorytmy DDM oraz EDDM zazwyczaj wykrywają dryft dużo częściej niż algorytm ADWIN, który zaś wykrywa dryft zauważalnie częściej od algorytmu Page Hinkley.
    \item Niezależnie od wybranych parametrów, w przypadku gdy klasyfikator cechuje się bezbłędną klasyfikacją, żaden z algorytmów nie wykrywa dryftów. W przypadku bardzo małych zmian (np. 0.025\%) algorytm DDM potrafi wykryć dryft, przez co jest on bardzo wrażliwy na pomyłki przy klasyfikacji.
    \item Naiwny klasyfikator Bayesa:
    \begin{itemize}
        \item Algorytm Page Hinkley wykrył dryft wyłącznie przy zastosowaniu miary czułości.
        \item Poza DDM algorytmy przy mierze precyzji bardzo rzadko wykrywają dryft.
        \item DDM sprawdził się w wykrywaniu spadków specyficzności -- wykres charakteryzuje się skokowymi zmianami, z którymi algorytm, wedle definicji, dobrze sobie radzi.
    \end{itemize}
    \item Maszyna wektorów nośnych:
    \begin{itemize}
        \item Parametr regularyzacji nie wpływa na wykrycie dryftu w przypadku jądra radialnego - dla regularyzacji 0,1 algorytm zachowuje się podobnie także w przypadku jądra wielomianego, dopiero przy regularyzacji 1 widzimy zmiany w miejsach wykrycia drfytu.
        \item Algorytm DDM dla miary dokładności wykrywa dryft jedynie na początku w przypadku większości parametrów przebiegu  klasyfikacji.
        \item Algorytm EDDM dla miar dokładności oraz precyzji wykrywa dryft zbyt często, zaś dla miary czułości - wcale.
        \item Algorytm ADWIN przy mierze specyficzności wykrywa dryft dużo częściej niż w inne algorytmy, co nie zdarza się w przypadku innych klasyfikatorów.
    \end{itemize}
    \item Klasyfikator k-najbliższych sąsiadów:
    \begin{itemize}
        \item Liczba najbliższych sąsiadów nie ma znaczącego wpływu na wykrycie dryftu dla większość algorytmów.
        \item Algorytm DDM dużo częściej wykrywa dryft dla 2 sąsiadów w przypadku miar dokładności oraz precyzji.
        \item Algorytm EDDM zwraca bardzo dużo punktów potencjalnych dryftów, natomiast w żaden sposób nie są one odzwierciedlane na wykresie. Może to świadczyć o tym, że nie jest on optymalnym wyborem dla tego klasyfikatora.
    \end{itemize}
    \item Sztuczna sieć neuronowa:
    \begin{itemize}
        \item Algorytmy wykrywają zauważalnie rzadziej dryft dla 10 warstw ukrytych oraz 1000 iteracji. Konfiguracja ta wykazała najlepsze wyniki we wszystkich wskaźnikach, dlatego brak dryftu jest tu logicznym efektem.
        \item Algorytm DDM na Rys. \ref{nn10-1000} pokazowo prezentuje teoretyczne działanie algorytmów detekcji dryftu
        \item Dla 1 warstwy ukrytej 1000 iteracji okazało się być zbyt dużą liczbą -- wystąpiło tutaj zjawisko przeuczenia, stąd skrajne wartości czułości i specyficzności. 
    \end{itemize}
    \item Sieć neuronowa okazała się najtrafniejszą metodą klasyfikacji dla wybranego przez nas zbioru danych.
    \item Nie da się jednoznacznie wybrać najlepszej metody detekcji dryftu. Każdy z nich sprawdzał się dobrze w określonych konfiguracjach. Dobór metody wydaje się więc być uzależniony od wybranego klasyfikatora oraz użytej miary.
   
\end{enumerate}

% \newpage
% \nocite{*}
% \begin{thebibliography}{}
% \end{thebibliography}

\end{document}
